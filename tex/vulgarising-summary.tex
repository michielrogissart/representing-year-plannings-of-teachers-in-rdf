\noindent A lot of data is created on the web everyday: people posting what they did, writing reviews or making their calendar for the coming week.
All this data is stored in `relational databases' which reside on computers often owned by large corporations.
A concept represented by this data, is called a resource. This can be anything from a tweet on Twitter, a laptop sold by a webshop or even a person.
When you create an account on a website, you become a resource, represented by an entry in a relational database of some company.
When you make different accounts on different websites, all these different entries represent the same resource: you.
This is because relational databases are big, structured chunks of memory in a computer.
Being `represented by an entry in a relational database', actually means being represented by some ones and zeros stored on a computer owned by someone else.\\ \\
Linked data works fundamentally different: rather than representing a resource with an entry in a relational database, it is represented by a URL.
Instead of chosing multiple usernames for multiple companies, you choose one URL for all companies. For example https://www.somedomain.com/myname, or https://firstname.lastname.com.\\
Again, everything can be represented by a url: people, books, numbers, items for sale...
Because URLs can be shared and referenced by anyone, data can be linked together.
If a laptop manufacturer provides a URL for a certain model, reviewing websites can refer to it.
The manufacturer can then write code to find all reviews referencing that particular model.\\
The link that was made, in this case between a laptop and a review, is semantic: it represents a real world relation. Namely that of someone having an opinion about a laptop.
These semantic relations (which are mostly standardized) make it easier to get complex insights in the data.\\ \\
In this paper, the advantages of linking data and retrieving complex insights are exploited in an educational context.
A data model is proposed for representing year plannings of teachers.
This would allow teachers to have insights in their lesson throughout the year with minimal input from themselves.
Because of decentralization, teachers can use data from different sources to plan their year optimally.
They could use data published by the governement to keep track of the competencies they need to cover, data published by publisher about the text books they use or even lesson preparations of colleagues.\\ \\
The model was created based on interviews conducted with teachers. This provided insights in how teachers maintain their year plannings now, and what information they're lacking about them.\\
Lessons here are considerd as a sequence of lesson phases, providing as much precision as desired.
These lesson phases can be annotated with data like the competencies they teach, which pages in which book are covered and which exercises are made.
Also links with educational taxonomies like the taxonomy of Bloom and data about the didactical method can be added. This enables teachers to offer enough variation throughout the year.\\
Not only teachers are data producers in this model. Another imporant actor are the publishers publishing text books.
They can publish data about the structure of their books: which chapters cover which competencies, which chapters belong together in a section, which exercises belong to which chapter...
Also exercises can be annotated: the difficulty, link with educational taxonomy, exercise types...\\ \\

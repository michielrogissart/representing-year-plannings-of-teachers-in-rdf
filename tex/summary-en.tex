\chapter*{Summary}
\noindent One of the advantages of linked data is that it allows insightful information about data with less complex queries than relation databases.
This can be exploited in the educational domain: teachers have to keep a lot in mind when planning lessons, so insightful data could help them plan their year better.
Tools exist that offer similar functionality, but this is rather limited due to their centralized nature and use of relational databases.\\
In this thesis, a data model is proposed to model year plannings in a linked data approach.\\
This model can later be used to develop tools helping teachers keep track of their year plannings using linked data.\\
Because of the decentralized nature of linked data, teachers can consume data produced by different sources.
For example, a publisher can publish metadata of their text books, or educational organizations can publish lesson ideas.\\ \\
Following the guidelines of the eXtreme Design methodology, teachers were interviewed to find out which tools they use, and what they lack.\\
The model was then developed following an iterative process of creating competency quetions, augmenting the model and constructing SPARQL queries answering these questions.
Data was created to test the model and queries.\\ 
When developing this model, granularity is important, as well as data consistency and the reuse of existing ontologies. This ensures a durable result.\\ \\
All competency questions have corresponding SPARQL queries, indicating that this is a working model that offers the functionality extracted from the interviews.
All data, competency questions and scripts creating the data, can be found in the GitHub repository of this thesis.

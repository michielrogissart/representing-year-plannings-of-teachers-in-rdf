\chapter*{Samenvatting}
\noindent Een van de voordelen van gelinkte data is dat het inzichtelijke informatie kan aanbieden met minder complexe operaties dan relationele databanken.
Dit kan worden benut in het onderwijs: leraren moeten veel in gedachten houden bij het plannen van lessen, dus inzichtelijke gegevens kunnen hen helpen hun jaar beter te plannen.
Er bestaan applicaties die vergelijkbare functionaliteiten aanbieden, maar deze zijn eerder beperkt vanwege hun gecentraliseerde karakter en het gebruik van relationele databases.\\
In dit proefschrift wordt een datamodel voorgesteld om jaarplannen te modelleren vanuit een gelinkte data benadering.\\
Dit model kan later worden gebruikt om applicaties te ontwikkelen waarmee leerkrachten hun jaarplannen kunnen bijhouden met behulp van gelinkte data.\\
Door het gedecentraliseerde karakter van gelinkte data, kunnen leerkrachten data gebruiken die door verschillende bronnen worden aangeboden.
Een uitgever kan bijvoorbeeld metadata van zijn studieboeken publiceren, of onderwijsorganisaties kunnen lesideeën publiceren.\\ \\
Conform de richtlijnen van de `eXtreme Design'-methodologie werden leerkrachten geïnterviewd om erachter te komen welke applicaties ze gebruiken en wat ze missen.\\
Het model is vervolgens ontwikkeld in een iteratief proces van het opstellen van competentievragen, het uitbreiden van het model en het construeren van SPARQL-query's die deze vragen beantwoorden.
Er is data gegenereerd om het model en de query's te testen.\\
Bij het ontwikkelen van dit model is granulariteit belangrijk, evenals dataconsistentie en het hergebruik van bestaande ontologieën. Dit zorgt voor een duurzaam resultaat.\\ \\
Alle competentievragen hebben bijbehorende SPARQL-query's, wat aangeeft dat dit een werkend model is dat de functionaliteit biedt die uit de interviews is gehaald.
Alle data, competentievragen en de code gebruikt om de gegevens te creëren, zijn te vinden in de GitHub-repository van deze thesis.